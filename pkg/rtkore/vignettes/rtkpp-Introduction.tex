%\VignetteIndexEntry{Short Introduction to rtkore}

\documentclass[a4paper,10pt]{article}

\usepackage[utf8]{inputenc}
\usepackage[T1]{fontenc}

\usepackage{amsmath,amssymb,amsthm}
\usepackage[english]{babel}
\usepackage{graphicx}
\usepackage{geometry}
\geometry{top=3cm, bottom=3cm, left=2cm, right=2cm}

\usepackage{url}

%-------------------------
% preamble for nice lsitings and notes

%-------------------
% Useful packages
\usepackage[utf8]{inputenc}
\usepackage[T1]{fontenc}

\usepackage{amsmath,amssymb,amsthm}
\usepackage[english]{babel}
\usepackage{graphicx}
\usepackage{geometry}
\usepackage{float}

\usepackage{url}
\usepackage{hyperref}
\usepackage{Sweave}

%-------------------
% Useful macros
\newcommand{\Rcpp}{{\tt Rcpp}} %
\newcommand{\rtkpp}{{\tt rtkpp}} %
\newcommand{\rtkore}{{\tt rtkore}} %
\newcommand{\stkpp}{{\tt STK++}} %
\newcommand{\Cpp}{{\tt C++}}

\newcommand{\Esp}[1]{\mathbb{E}\left[#1\right]}
\newcommand{\Var}[1]{\mathrm{Var}\left[#1\right]}
\newcommand{\E}[1]{\mathbb{E}\left[#1\right]}
\newcommand{\V}[1]{\mathrm{Var}\left[#1\right]}
\newcommand{\Loi}[1]{\mathcal{L}\left(#1\right)}

%----- Sets : integers, reals, complexes -----
\newcommand{\Nn}{\mathbb{N}} \newcommand{\N}{\mathbb{N}}
\newcommand{\Zz}{\mathbb{Z}} \newcommand{\Z}{\mathbb{Z}}
\newcommand{\Qq}{\mathbb{Q}} \newcommand{\Q}{\mathbb{Q}}
\newcommand{\Rr}{\mathbb{R}} \newcommand{\R}{\mathbb{R}}
\newcommand{\Cc}{\mathbb{C}}

%-------------------
% Multiple lines in a tabular
\newcommand{\vcell}[2][c]{%
\begin{tabular}[#1]{@{}l@{}}#2\end{tabular}}

%-------------------
% Nice notes and warnings
\usepackage{mdframed} % Add easy frames to paragraphs
\usepackage{xcolor}
\usepackage{xparse} % Add support for \NewDocumentEnvironment
\definecolor{graylight}{cmyk}{.30,0,0,.67} % define color using xcolor syntax
\definecolor{redlight}{rgb}{1,.2,.2}       % define color using xcolor syntax

\newmdenv[ % Define mdframe settings and store as leftrule
  linecolor=graylight,
  topline=false,
  bottomline=false,
  rightline=false,
  skipabove=\topsep,
  skipbelow=\topsep
]{leftrule}

\newmdenv[ % Define mdframe settings and store as leftrule
  linecolor=redlight,
  topline=false,
  bottomline=false,
  rightline=false,
  skipabove=\topsep,
  skipbelow=\topsep
]{redleftrule}

\NewDocumentEnvironment{note}{O{\textbf{Note:}}} % Define example environment
{\begin{leftrule}\noindent\textcolor{graylight}{#1}\par}
{\end{leftrule}}

\NewDocumentEnvironment{warning}{O{\textbf{Warning:}}} % Define warning environment
% environment
{\begin{redleftrule}\noindent\textcolor{redlight}{#1}\par}
{\end{redleftrule}}

%-------------------
% nice listing
\usepackage{listings}
\usepackage{caption}

\lstdefinestyle{customcpp}{
  language=C++,
  belowcaptionskip=1\baselineskip,
  breaklines=true,
  basicstyle=\ttfamily\scriptsize,
  frame=single,
  xleftmargin=\parindent,
  showstringspaces=false,
  keywordstyle=\color{red},
  commentstyle=\ttfamily\scriptsize\color{green!40!black},
  identifierstyle=\color{blue},
  stringstyle=\color{orange},
}
\lstdefinestyle{inlinecpp}{
  language=C++,
  belowcaptionskip=1\baselineskip,
  basicstyle=\ttfamily,
  frame=single,
  xleftmargin=\parindent,
  showstringspaces=false,
  keywordstyle=\color{red},
  commentstyle=\ttfamily\color{green!40!black},
  identifierstyle=\color{blue},
  stringstyle=\color{orange},
}


\DeclareCaptionFormat{listing}
{
  \colorbox[cmyk]{0.43, 0.35, 0.35, 0.01}
  { \parbox{\textwidth}{\hspace{15pt}#1#2#3}}
}
\captionsetup[lstlisting]{ format=listing, %labelfont=white, textfont=white
                         , singlelinecheck=false, margin=-0.2cm
                         , font={bf,footnotesize} }

\newcommand{\includelstlisting}[1]{\lstinputlisting[style=customcpp]{#1}}
\newcommand{\code}[1]{\lstinline[style=inlinecpp]@#1@}
\newcommand{\ttcode}[1]{{\ttfamily\scriptsize #1}}
% end nice listing
%---------------------


%
%-------------------------


\usepackage{Sweave}
%% need no \usepackage{Sweave.sty}
% Title Page
\title{ rtkore: R and STK++ Integration using Rcpp}
\author{Serge Iovleff}
\date{\today}

% start documentation
\begin{document}
\input{rtkpp-Introduction-concordance}

\maketitle
\begin{abstract}
This vignette gives some hints about the usage of the \rtkore{} (successor of
the \rtkpp{}) package. It explains shortly how to wrap R vectors and matrices
into \stkpp{} structures. It gives also an example of Makevars for linking an R
package with \rtkore{}. More informations can be found in the other vignettes
coming with the package about the functionnalities furnished by the \stkpp{} library.
\end{abstract}

\section{Introduction}

\stkpp{} is a versatile, fast, reliable and elegant collection of \texttt{C++}
classes for statistics, clustering, linear algebra (using native methods or
Lapack\cite{Lapack}), arrays (with an Eigen-like API \cite{JSS:RcppEigen}),
regression, dimension reduction, etc. Some functionalities provided by the
library are available in the \texttt{R} environment as \texttt{R} functions or
distributed as R packages (\texttt{MixAll} \cite{MixAll} and \texttt{HDPenReg}
\cite{HDPenReg} among others).

The \rtkore{} package provides a subset of the \stkpp{}
library and is only composed of templated classes and inlined functions.
The \rtkpp{} package is also available and provides the header files composing
the whole \stkpp{} library. Theses packages furnish implementations of
\texttt{Rcpp::as} and \texttt{Rcpp::wrap} for the \texttt{C++} classes defined
in \stkpp{}. In this sense it is similar to the \texttt{RcppEigen}
\cite{CRAN:RcppEigen,JSS:RcppEigen} and \texttt{RcppArmadillo}
\cite{CRAN:RcppArmadillo} packages.

The current version of the stk++ library is given below
\begin{Schunk}
\begin{Sinput}
> .Call("stk_version", FALSE, PACKAGE="rtkore")
\end{Sinput}
\begin{Soutput}
major minor patch 
    0     9     7 
\end{Soutput}
\end{Schunk}

\section{Wrapping \texttt{R} data with \stkpp{} arrays}

rtkore proposes two objects in order to facilitate data transfer
\begin{lstlisting}[style=customcpp]
typename RVector<Type>;
typename RMatrix<Type>;
\end{lstlisting}
\texttt{Rcpp} facilitates conversion of objects from \texttt{R} to
\texttt{C++} through the templated functions \texttt{Rcpp::as}.
The function \texttt{Rcpp::as} is re-implemented in \stkpp{} but
it is not strictly necessary to use it. You can rather use this kind
of code
\begin{lstlisting}[style=customcpp]
SEXP myFunction(SEXP data)
{
  STK::RMatrix<double> mat(data); // if data is not a matrix, Rcpp will throw an exception
  // ....
  // wrap a Rcpp matrix in a STK++ matrix
  Rcpp::NumericMatrix rmat(100,20);
  STK::RMatrix<double> mat(rmat);      // wrap
  // Constructor with given dimension
  RMatrix<double> myData(100, 20);
  // Copy constructor
  RMatrix<double> myCopy(mat)
}
\end{lstlisting}
The template class \texttt{STK::RMatrix} wraps a Rcpp matrix which itself
wrap the \texttt{R} \verb+SEXP+ structure. You can access directly (and
eventually modify) the \texttt{R} data in your application like
an usual \stkpp{} array.

The second template class you can use is \texttt{STK::RVector} which
allows to wrap \texttt{Rcpp::Vector} class.

\section{Converting \stkpp{} arrays and expressions to \texttt{R} data}

\texttt{Rcpp} facilitates data conversion from \texttt{C++} to \texttt{R}
through  \texttt{Rcpp::wrap}. This function is extended by \rtkore{} for \stkpp{}
arrays and vectors.

The following example is taken from the \texttt{STK::ClusterLauncher} class  (MixAll
package)
\begin{lstlisting}[style=customcpp]
  Array2D<Real> mean(K, nbVariable), sigma(K, nbVariable);
  // get estimated parameters
  // ....
  // and save them
  NumericVector m_mean  = Rcpp::wrap(mean);
  NumericVector m_sigma = Rcpp::wrap(sigma);
\end{lstlisting}

Note that the \texttt{Rcpp::wrap} is rather limited in its usage and if you
need, for example, to convert expression rather than arrays then you can use the
\texttt{STK::wrap} function (see example below).

\section{Using \rtkore{} random number generators}

All the random numbers of R are interfaced in \rtkore{}. You can used them as
\stkpp{} random number generators like in the following example

\begin{lstlisting}[style=customcpp]
RcppExport SEXP fastBetaRand( SEXP n, SEXP alpha, SEXP beta)
{
  BEGIN_RCPP;
  // create a STK++ RVector
  STK::RVector<double> tab(Rcpp::as<int>(n));
  // Create a Beta distribution function with alpha and beta as parameters
  STK::Law::Beta law(Rcpp::as<double>(alpha), Rcpp::as<double>(beta));
  // fill tab with random numbers
  tab.rand(law);
  // return the wrapped Rcpp vector
  return tab.vector();
  END_RCPP;
}
\end{lstlisting}


\section{Linking with \rtkore{}}

At the R level, you have to add the \texttt{LinkingTo: rtkore,Rcpp} line in the
\verb+DESCRIPTION+ file.

\noindent At the C++ level, the only thing to do is to include the header file
\begin{lstlisting}[style=customcpp]
// Rcpp.h will be include by rtkore
#include <RTKpp.h>
\end{lstlisting}
in the C++ code.

When compiling the sources, you indicate the location of the stk++ library using
\verb+rtkore:::CxxFlags()+, \verb+rtkore:::CppFlags()+ and
\verb+rtkore:::LdFlags()+ in the \texttt{src/Makevars} file.

A minimal Makevars would look like
\begin{verbatim}
PKG_CXXFLAGS = `${R_HOME}/bin/Rscript -e "rtkore:::CxxFlags()"`
PKG_CPPFLAGS = `${R_HOME}/bin/Rscript -e "rtkore:::CppFlags()"` $(SHLIB_OPENMP_CXXFLAGS)
PKG_LIBS     = `$(R_HOME)/bin/Rscript -e "rtkore:::LdFlags()"` \
                $(SHLIB_OPENMP_CFLAGS) $(LAPACK_LIBS) $(BLAS_LIBS) $(FLIBS)
\end{verbatim}

If you are building a package with a lot of cpp files, you may find
convenient to locate your sources in a separate directory. Hereafter we give an
example of a Makevars you can modify at your convenience in order to handle
this situation.
\begin{verbatim}
#-----------------------------------------------------------------------
# Purpose:  Makevars for the R packages using rtkore (stk++)
#-----------------------------------------------------------------------
PKGNAME   = NAME_OF_YOUR_SRC  # for example MyPackage
PKGDIR    = PATH_TO_YOUR_SRC  # for example ./MyPackage
PKGLIBDIR = $(PKGDIR)/lib     # ./MyPackage/lib
PKGLIB    = $(PKGLIBDIR)/lib$(PKGNAME).a # ./MyPackage/lib/libMyPackage.a

## Use the R_HOME indirection to support installations of multiple R version.
PKG_CXXFLAGS = `${R_HOME}/bin/Rscript -e "rtkore:::CxxFlags()"`
PKG_CPPFLAGS = `${R_HOME}/bin/Rscript -e "rtkore:::CppFlags()"` \
                $(SHLIB_OPENMP_CXXFLAGS)

## We link the source in the src/ directory with the stkpp library and libMyPackage.a
## use $(SHLIB_OPENMP_CFLAGS) as stkpp use openMP
## use $(LAPACK_LIBS) $(BLAS_LIBS) $(FLIBS) if you want to use lapack and/or stk++
## wrappers of lapack
PKG_LIBS = `$(R_HOME)/bin/Rscript -e "rtkore:::LdFlags()"` $(PKGLIB) \
					$(SHLIB_OPENMP_CFLAGS) \
          $(LAPACK_LIBS) $(BLAS_LIBS) $(FLIBS)

## Define any flags you may need for compiling your sources and export them
MY_CXXFLAGS = $(PKG_CXXFLAGS)
MY_CPPFLAGS = $(PKG_CPPFLAGS)

export

.PHONY: all pkglib

## $(SHLIB) is the usual default target that is built automatically from all source
## files in this directory. pkglib is an additional target for the package
## that will be found in $(PKGDIR).
all: $(SHLIB)
$(SHLIB): pkglib

## build the PKGLIB (lib$(PKGNAME).a)
pkglib:
	(cd $(PKGDIR) && $(MAKE)  all)
	(cd $(PKGDIR) && $(MAKE) clean)
\end{verbatim}

\section{An example}

The package countMissings can be downloaded at the
\url{http://sourceforge.net/projects/stkpp/files/R%20packages/countMissings_1.0.tar.gz/download}
url. It is basically composed of one R-script file (countNA.R) and one C++ file
(countNA.cpp).

Given a R matrix, you will get a list composed of two vectors
constaining respectively the number of missing values in each rows and the
number of missing values in each columns of the R matrix.

The R-script \texttt{countNA.R} is essentially
\begin{lstlisting}[style=customcpp]
countNA <- function(data)
{
  if (!is.matrix(data)) { stop("in countNA, data must be a matrix.")}
  .Call("countNA", data, PACKAGE = "countMissings")
}
\end{lstlisting}
and the C++ files is
\begin{lstlisting}[style=customcpp]
#include "RTKpp.h"
RcppExport SEXP countNA( SEXP r_matrix)
{
  BEGIN_RCPP
  STK::RMatrix<double> m_data(r_matrix);
  // use STK::wrap function (Rcpp::wrap function will not work)
  return Rcpp::List::create( Rcpp::Named("rows")= STK::wrap(STK::countByRow(m_data.isNA()))
                           , Rcpp::Named("cols")= STK::wrap(STK::count(m_data.isNA()))
                           );
  END_RCPP
}
\end{lstlisting}


\bibliographystyle{plain}
\bibliography{rtkore}

\end{document}

%\VignetteIndexEntry{STK++ Utilities, Reference Guide}

\documentclass[a4paper,10pt]{article}

%-------------------------
% preamble for nice lsitings and notes

%-------------------
% Useful packages
\usepackage[utf8]{inputenc}
\usepackage[T1]{fontenc}

\usepackage{amsmath,amssymb,amsthm}
\usepackage[english]{babel}
\usepackage{graphicx}
\usepackage{geometry}
\usepackage{float}

\usepackage{url}
\usepackage{hyperref}
\usepackage{Sweave}

%-------------------
% Useful macros
\newcommand{\Rcpp}{{\tt Rcpp}} %
\newcommand{\rtkpp}{{\tt rtkpp}} %
\newcommand{\rtkore}{{\tt rtkore}} %
\newcommand{\stkpp}{{\tt STK++}} %
\newcommand{\Cpp}{{\tt C++}}

\newcommand{\Esp}[1]{\mathbb{E}\left[#1\right]}
\newcommand{\Var}[1]{\mathrm{Var}\left[#1\right]}
\newcommand{\E}[1]{\mathbb{E}\left[#1\right]}
\newcommand{\V}[1]{\mathrm{Var}\left[#1\right]}
\newcommand{\Loi}[1]{\mathcal{L}\left(#1\right)}

%----- Sets : integers, reals, complexes -----
\newcommand{\Nn}{\mathbb{N}} \newcommand{\N}{\mathbb{N}}
\newcommand{\Zz}{\mathbb{Z}} \newcommand{\Z}{\mathbb{Z}}
\newcommand{\Qq}{\mathbb{Q}} \newcommand{\Q}{\mathbb{Q}}
\newcommand{\Rr}{\mathbb{R}} \newcommand{\R}{\mathbb{R}}
\newcommand{\Cc}{\mathbb{C}}

%-------------------
% Multiple lines in a tabular
\newcommand{\vcell}[2][c]{%
\begin{tabular}[#1]{@{}l@{}}#2\end{tabular}}

%-------------------
% Nice notes and warnings
\usepackage{mdframed} % Add easy frames to paragraphs
\usepackage{xcolor}
\usepackage{xparse} % Add support for \NewDocumentEnvironment
\definecolor{graylight}{cmyk}{.30,0,0,.67} % define color using xcolor syntax
\definecolor{redlight}{rgb}{1,.2,.2}       % define color using xcolor syntax

\newmdenv[ % Define mdframe settings and store as leftrule
  linecolor=graylight,
  topline=false,
  bottomline=false,
  rightline=false,
  skipabove=\topsep,
  skipbelow=\topsep
]{leftrule}

\newmdenv[ % Define mdframe settings and store as leftrule
  linecolor=redlight,
  topline=false,
  bottomline=false,
  rightline=false,
  skipabove=\topsep,
  skipbelow=\topsep
]{redleftrule}

\NewDocumentEnvironment{note}{O{\textbf{Note:}}} % Define example environment
{\begin{leftrule}\noindent\textcolor{graylight}{#1}\par}
{\end{leftrule}}

\NewDocumentEnvironment{warning}{O{\textbf{Warning:}}} % Define warning environment
% environment
{\begin{redleftrule}\noindent\textcolor{redlight}{#1}\par}
{\end{redleftrule}}

%-------------------
% nice listing
\usepackage{listings}
\usepackage{caption}

\lstdefinestyle{customcpp}{
  language=C++,
  belowcaptionskip=1\baselineskip,
  breaklines=true,
  basicstyle=\ttfamily\scriptsize,
  frame=single,
  xleftmargin=\parindent,
  showstringspaces=false,
  keywordstyle=\color{red},
  commentstyle=\ttfamily\scriptsize\color{green!40!black},
  identifierstyle=\color{blue},
  stringstyle=\color{orange},
}
\lstdefinestyle{inlinecpp}{
  language=C++,
  belowcaptionskip=1\baselineskip,
  basicstyle=\ttfamily,
  frame=single,
  xleftmargin=\parindent,
  showstringspaces=false,
  keywordstyle=\color{red},
  commentstyle=\ttfamily\color{green!40!black},
  identifierstyle=\color{blue},
  stringstyle=\color{orange},
}


\DeclareCaptionFormat{listing}
{
  \colorbox[cmyk]{0.43, 0.35, 0.35, 0.01}
  { \parbox{\textwidth}{\hspace{15pt}#1#2#3}}
}
\captionsetup[lstlisting]{ format=listing, %labelfont=white, textfont=white
                         , singlelinecheck=false, margin=-0.2cm
                         , font={bf,footnotesize} }

\newcommand{\includelstlisting}[1]{\lstinputlisting[style=customcpp]{#1}}
\newcommand{\code}[1]{\lstinline[style=inlinecpp]@#1@}
\newcommand{\ttcode}[1]{{\ttfamily\scriptsize #1}}
% end nice listing
%---------------------



\geometry{top=2cm, bottom=2cm, left=1cm, right=1cm}

%% need no \usepackage{Sweave.sty}
% Title Page
\title{STK++ Utilities reference guide}
\author{Serge Iovleff}
\date{\today}

% start documentation
\begin{document}
\Sconcordance{concordance:rtkpp-Utilities.tex:rtkpp-Utilities.Rnw:%
1 10 1 1 6 570 1}


\maketitle

\begin{abstract}
This reference guide gives a general review of the capabilities offered
by the \stkpp{} library. The library is divided in various \emph{projects}.
The "Arrays" project is described in detail in the vignette "STK++ Arrays,
User Guide" (\cite{rtkore-Arrays}) and the quick reference guide.
This vignette focus on the "STatistiK" project and the "Algebra" project.
\end{abstract}

\tableofcontents

\section{Statistical functors, methods and functions (STatistiK project)}
\label{sec:STatistiK}

This section describe the main features provided by the STatistiK project. Mainly
\begin{enumerate}
\item the probability classes (\ref{subsec:prob}),
\item the descriptive statistical methods,
\item the utilities related methods.
\end{enumerate}

\subsection{Probabilities }
\label{subsec:prob}
All the probabilities handled by R are available in \rtkore{}. In the stand-alone \stkpp{} library, only a
subset of theses probabilities are implemented. Probability distribution classes are defined in the
\code{namespace Law} and can be used as in this example

\begin{minipage}[t]{0.66\textwidth}
\lstinputlisting[style=customcpp,caption=Example]{programs/tutoProbability.cpp}
\end{minipage}
\hspace{0.2cm}
\begin{minipage}[t]{0.33\textwidth}
\addtocounter{lstlisting}{-1}
\lstinputlisting[style=customcpp,caption=Output]{programs/tutoProbability.out}
\end{minipage}

All probability distribution classes have a similar prototype like the one given below
\begin{lstlisting}[style=customcpp,caption=Prototype of probability distribution class (example taken from Cauchy class)]
class Cauchy : public IUnivLaw<Real>
{
  public:
    Cauchy( Real const& mu=0, Real const& scale=1);
    virtual ~Cauchy() {}
    Real const& mu() const;
    Real const& scale() const;
    void setMu( Real const& mu);
    void setScale( Real const& scale);
    virtual Real rand() const;
    virtual Real pdf( Real const& x) const;
    virtual Real lpdf( Real const& x) const;
    virtual Real cdf( Real const& t) const;
    virtual Real icdf( Real const& p) const;
    static Real rand( Real const& mu, Real const& scale);
    static Real pdf( Real const& x, Real const& mu, Real const& scale);
    static Real lpdf( Real const& x, Real const& mu, Real const& scale);
    static Real cdf( Real const& t, Real const& mu, Real const& scale);
    static Real icdf( Real const& p, Real const& mu, Real const& scale);
  protected:
    Real mu_;
    Real scale_;
}
\end{lstlisting}
If $f$ denote the density of some probability distribution function (pdf) on $\R$,
the methods have the following meaning
\begin{enumerate}
\item \code{pdf(x)} return the value $f(x)$,
\item \code{lpdf(x)} return the value $\log(f(x))$,
\item \code{rand()} return a a realization of the density $f$,
\item \code{cdf(t)} return the cumulative distribution function value $F(t)=\int_{-\infty}^t f(x) dx$
\item \code{icdf(p)} return the inverse cumulative distribution function value $F^{-1}(p)$.
\end{enumerate}

The table \ref{tab:prob} gives the list of available probability distribution.
\begin{table}[htb]
\begin{tabular}{|l|l|l|l|}
\hline
Name             & Constructor                                & R functions     & Notes\\
\hline
Bernoulli        & \ttcode{Law::Bernouilli(p) }          & -               & \\
\hline
Binomial         & \ttcode{Law::Binomial(n,p)}           & \ttcode{*binom} & \\
\hline
Beta             & \ttcode{Law::Beta(alpha,beta)}        & \ttcode{*beta} & \\
\hline
Categorical      & \ttcode{Law::Categorical(p)}          & -               & p can be any \stkpp{} vector \\
\hline
Cauchy           & \ttcode{Law::Cauchy(m,s)}             & \ttcode{*cauchy}& \\
\hline
ChiSquared       & \ttcode{Law::ChiSquared(n)}           & \ttcode{*chisq} & \\
\hline
Exponential      & \ttcode{Law::Exponential(lambda)}     & \ttcode{*exp}   & Parameterization $\frac{e^{-x/\lambda}}{\lambda}$ \\
\hline
FisherSnedecor   & \ttcode{Law::FisherSnedecor(df1,df2)} &  \ttcode{*f}    & \\
\hline
Gamma            & \ttcode{Law::Gamma(a,b)}              & \ttcode{*gamma} & Parameterization $\frac{x^{a-1}}{\beta^a\Gamma(a)} e^{-x/\beta}$ \\
\hline
Geometric        & \ttcode{Law::Geometric(p)}            & \ttcode{*geom}  &  \\
\hline
HyperGeometric   & \ttcode{Law::HyperGeometric(m,n,k)}   & \ttcode{*hyper} & \\
\hline
Logistic         & \ttcode{Law::Logistic(mu,scale)}      & \ttcode{*logis} & \\
\hline
LogNormal        & \ttcode{Law::LogNormal(mulog,sdlog)}  & \ttcode{*lnorm} & \\
\hline
NegativeBinomial & \ttcode{Law::NegativeBinomial(size,prob,mu)}& \ttcode{*nbinom}&\\
\hline
Normal           & \ttcode{Law::Normal(mu,sigma)}        & \ttcode{*norm}   &\\
\hline
Poisson          & \ttcode{Law::Poisson(lambda)}         & \ttcode{*poiss}  & \\
\hline
Student          & \ttcode{Law::Student(df)}             & \ttcode{*t}      &\\
\hline
Uniform          & \ttcode{Law::Uniform(a,b)}            & \ttcode{*unif}   & \\
\hline
UniformDiscrete  & \ttcode{Law::UniformDiscrete(a,b)}    & -                & \\
\hline
Weibull          & \ttcode{Law::Weibull(a)}              & \ttcode{*weibull} & \\
\hline
\end{tabular}
\caption{List of the available probability distribution in \rtkore{}}
\label{tab:prob}
\end{table}

\subsection{Statistical Methods and global functions}

\stkpp{} provides a lot of methods, functions and functors in order to compute
usual statistics.

\subsubsection{Methods}
Let $m$ be any kind of array (square, vector, point, etc...). it is possible to
compute the $\min$, $\max$, mean, variance of the elements. These computations
can be safe (i.e. discarding N.A. and infinite values) or unsafe and weighted

\begin{table}[H]
\begin{tabular}{|l|l|l|l|}
\hline
Method                  & weigthed version          & safe versions  & Notes\\
\hline
\ttcode{m.norm()}       & \ttcode{m.wnorm(w)}       & \ttcode{m.normSafe(); m.wnormSafe(w)}       & $\sqrt{\sum |m_{ij}|^2}$ \\
\hline
\ttcode{m.norm2()}      & \ttcode{m.wnorm2(w)}      & \ttcode{m.norm2Safe(); m.wnorm2Safe(w)}     & $\sum |m_{ij}|^2$  \\
\hline
\ttcode{m.normInf()}    & \ttcode{m.wnormInf(w)}    & \ttcode{m.normInfSafe(); m.wnormInfSafe(w)} & $\sup |m_{ij}|$\\
\hline
\ttcode{m.sum()}        & \ttcode{m.wsum(w)}        & \ttcode{m.sumSafe(); m.wsumSafe(w)}         & $\sum m_{ij}$ \\
\hline
\ttcode{m.mean()}       & \ttcode{m.wmean(w)}       & \ttcode{m.meanSafe(); m.wmeanSafe(w)}        & $\frac{1}{n}\sum m_{ij}$\\
\hline
\ttcode{m.variance()}   & \ttcode{m.wvariance(w)}   & \ttcode{m.varianceSafe(); m.wvarianceSafe(w)} & $\frac{1}{n}\sum (m_{ij}-\bar{m})^2$\\
\hline
\ttcode{m.variance(mu)} & \ttcode{m.wvariance(mu,w)}& \ttcode{m.varianceSafe(mu); m.wvarianceSafe(mu,w)}& $\frac{1}{n}\sum (m_{ij}-\mu)^2$\\
\hline
\end{tabular}
\caption{List of the available statistical methods for the arrays. $n$ represents the number of elements of $m$. For safe
versions, $n$ represents the number of available observations in $m$.}
\end{table}

\subsubsection{Statistical functions}

For two dimensional arrays, there exists global functions allowing to compute
the usual statistics by column or by row. By default all global functions are
computing the statistics columns by columns. For example, if $m$ is an array, \code{sum(m)}
return a row-vector of range \code{m.cols()} containing the sum of each columns.
The alias \code{sumByCol(m)} can also be used. The sum of each rows can be obtained
using the function \code{sumByow(m)} which return an array of range \code{m.rows()}.

These computations can be safe (i.e. discarding N.A. and infinite values) or unsafe
and/or weighted.

\begin{table}[H]
\begin{tabular}{|l|l|l|l|}
\hline
Function                 & weigthed version        & safe versions  \\
\hline
\ttcode{min(m)}          & \ttcode{min(m, w)}      &\ttcode{minSafe(m/*,w*/)} \\
\ttcode{minByCol(m)}     & \ttcode{minByCol(m, w)} &\ttcode{minSafeByCol(m/*,w*/)} \\
\ttcode{minByRow(m)}     & \ttcode{minByRow(m, w)} &\ttcode{minSafeByRow(m/*,w*/)} \\
\hline
\ttcode{max(m)}          & \ttcode{max(m, w)}      &\ttcode{maxSafe(m/*,w*/)} \\
\ttcode{maxByCol(m)}     & \ttcode{maxByCol(m, w)} &\ttcode{maxSafeByCol(m/*,w*/)} \\
\ttcode{maxByRow(m)}     & \ttcode{maxByRow(m, w)} &\ttcode{maxSafeByRow(m/*,w*/)} \\
\hline
\ttcode{sum(m)}          & \ttcode{sum(m, w)}      &\ttcode{sumSafe(m/*,w*/)} \\
\ttcode{sumByCol(m)}     & \ttcode{sumByCol(m, w)} &\ttcode{sumSafeByCol(m/*,w*/)} \\
\ttcode{sumByRow(m)}     & \ttcode{sumByRow(m, w)} &\ttcode{sumSafeByRow(m/*,w*/)} \\
\hline
\ttcode{mean(m)}          & \ttcode{mean(m, w)}      &\ttcode{meanSafe(m/*,w*/)}   \\
\ttcode{meanByCol(m)}     & \ttcode{meanByCol(m, w)} &\ttcode{meanSafeByCol(m/*,w*/)} \\
\ttcode{meanByRow(m)}     & \ttcode{meanByRow(m, w)} &\ttcode{meanSafeByRow(m/*,w*/)} \\
\hline
\ttcode{variance(m, unbiased)}          & \ttcode{variance(m, w, unbiased)}      &\ttcode{varianceSafe(m/*,w*/, unbiased)}      \\
\ttcode{varianceByCol(m, unbiased)}     & \ttcode{varianceByCol(m, w, unbiased)} &\ttcode{varianceSafeByCol(m/*,w*/, unbiased)} \\
\ttcode{varianceByRow(m, unbiased)}     & \ttcode{varianceByRow(m, w, unbiased)} &\ttcode{varianceSafeByRow(m/*,w*/, unbiased)} \\
\hline
\ttcode{varianceWithFixedMean(m, mu, unbiased)}       & \ttcode{variance*(m, w, mu, unbiased)}      &\ttcode{variance*Safe(m/*,w*/, mu, unbiased)}  \\
\ttcode{varianceWithFixedMeanByCol(m, mu, unbiased)}  & \ttcode{variance*ByCol(m, w, mu, unbiased)} &\ttcode{variance*SafeByCol(m/*,w*/, mu, unbiased)} \\
\ttcode{varianceWithFixedMeanByRow(m, mu, unbiased)}  & \ttcode{variance*ByRow(m, w, mu, unbiased)} & \ttcode{variance*SafeByRow(m/*,w*/, mu, unbiased)} \\
\hline
\end{tabular}
\caption{List of the available global statistical functions for arrays. $m$ is the array, $w$ the vector of weights. \code{unbiased}
is a Boolean to set to \code{true} if unbiased variance (divided by $n-1$) is desired. }
\end{table}

The covariance can be computed in two ways : using two vectors of same size, or using all
the columns (rows) of a two-dimensional array. In the first case the functions return
the value of the covariance, in the second case, they return a \code{CSquareArray}.

\begin{table}[H]
\begin{tabular}{|l|l|l|l|}
\hline
Function                               & weigthed version          & safe versions \\
\hline
\ttcode{covariance(v1, v2, unbiased)}  & \ttcode{covariance(v1, v2, w, unbiased)} &
\vcell{ \ttcode{covarianceSafe(v1, v2, unbiased)} \\ \ttcode{covarianceSafe(v1, v2, w, unbiased)} } \\
\hline
\ttcode{covarianceWithFixedMean(v1, v2, mean, unbiased)}  & \ttcode{covariance*(v1, v2, w, mean, unbiased)} &\ttcode{covariance*Safe(v1, v2, unbiased)} \\
\hline
\hline
\ttcode{covariance(m, unbiased)}       & \ttcode{covariance(m, w, unbiased)} & \\
\ttcode{covarianceByRow(m, unbiased)}       & \ttcode{covarianceByRow(m, w, unbiased)} & \\
\hline
\ttcode{covarianceWithFixedMean(m, mean, unbiased)}       & \ttcode{covariance*(m, w, mean, unbiased)} & \\
\ttcode{covarianceWithFixedMeanByRow(m, mean, unbiased)}       & \ttcode{covariance*ByRow(m, w, mean, unbiased)} & \\
\hline
\end{tabular}
\caption{List of the available covariance functions for vectors and arrays. $v_1$ and $v_2$ are vectors, $m$ is an array, $w$ a vector of weights. \code{unbiased}
is a Boolean to set to \code{true} if unbiased covariance (divided by $n-1$) is desired.
The first set of covariance functions return a scalar, the second set of covariance functions return a \code{CSquareArray}}
\end{table}

The following example illustrate the use of the covariance function:

\begin{minipage}[t]{0.66\textwidth}
\lstinputlisting[style=customcpp,caption=Example]{programs/tutoStatCovariance.cpp}
\end{minipage}
\hspace{0.2cm}
\begin{minipage}[t]{0.33\textwidth}
\addtocounter{lstlisting}{-1}
\lstinputlisting[style=customcpp,caption=Output]{programs/tutoStatCovariance.out}
\end{minipage}

\subsection{Miscellaneous statistical functions}

Given an array $m$, it is possible to center it, to standardize it and to perform the
reverse operations as illustrated in the following exemple

\begin{minipage}[t]{0.66\textwidth}
\lstinputlisting[style=customcpp,caption=Example]{programs/tutoStatTransform.cpp}
\end{minipage}
\hspace{0.2cm}
\begin{minipage}[t]{0.33\textwidth}
\addtocounter{lstlisting}{-1}
\lstinputlisting[style=customcpp,caption=Output]{programs/tutoStatTransform.out}
\end{minipage}

\begin{table}[H]
\begin{tabular}{|l|l|}
\hline
Function                 & weigthed version \\
\hline
\ttcode{center(m, mean)} & \ttcode{center(m, w, mean)} \\
\ttcode{centerByCol(m, mean)} & \ttcode{centerByCol(m, w, mean)} \\
\ttcode{centerByRow(m, mean)} & \ttcode{centerByRow(m, w, mean)} \\
\hline
\ttcode{standardize(m, std, unbiased)} & \ttcode{standardize(m, w, std, unbiased)} \\
\ttcode{standardizeByCol(m, std, unbiased)} & \ttcode{standardizeByCol(m, w, std, unbiased)} \\
\ttcode{standardizeByRow(m, std, unbiased)} & \ttcode{standardizeByRow(m, w, std, unbiased)} \\
\hline
\ttcode{standardize(m, mean, std, unbiased)} & \ttcode{standardize(m, w, mean, std, unbiased)} \\
\ttcode{standardizeByCol(m, mean, std, unbiased)} & \ttcode{standardizeByCol(m, w, mean, std, unbiased)} \\
\ttcode{standardizeByRow(m, mean, std, unbiased)} & \ttcode{standardizeByRow(m, w, mean, std, unbiased)} \\
\hline
\ttcode{uncenter(m, mean)}      &  \\
\ttcode{uncenterByCol(m, mean)} &  \\
\ttcode{uncenterByRow(m, mean)} &  \\
\hline
\ttcode{unstandardize(m, std)}      &  \\
\ttcode{unstandardizeByCol(m, std)} &  \\
\ttcode{unstandardizeByRow(m, std)} &  \\
\hline
\ttcode{unstandardize(m, mean, std)}      &  \\
\ttcode{unstandardizeByCol(m, mean, std)} &  \\
\ttcode{unstandardizeByRow(m, mean, std)} &  \\
\hline
\end{tabular}
\caption{List of the available utilites functions for centering and/or standardize an array.
$m$ is an array, $w$ a vector of weights. If used on the columns, \code{mean} and \code{std}
have to be points (row-vectors), if used by rows, \code{mean} and \code{std} have to be
vectors.\code{unbiased} is a boolean to set to \code{true} if unbiased covariance (divided by $n-1$)
is desired.}
\end{table}

\subsection{Computing factors}

Given a finite collection of object in a vector or an array/expression, it is possible to encode
it as factor using the classes \code{Stat::Factor} (for vectors) and \code{Stat::MultiFactor}
(for array). These classes are runners and you have to use the \code{run} method in order to
trigger the computation of the factors.

An exemple is given below

\begin{minipage}[t]{0.66\textwidth}
\lstinputlisting[style=customcpp,caption=Example]{programs/tutoStatFactors.cpp}
\end{minipage}
\hspace{0.2cm}
\begin{minipage}[t]{0.33\textwidth}
\addtocounter{lstlisting}{-1}
\lstinputlisting[style=customcpp,caption=Output]{programs/tutoStatFactors.out}
\end{minipage}

\section{Linear Algebra classes, methods and functions}

\stkpp{} basic linear operation as product, dot product, sum, multiplication
by a scalar,... are encoded in template expressions and optimized.

Since \stkpp{} version 0.9 and later, LAPACK library can be used as backends for dense matrix
matrix decompositions (QR, Svd, eigenvalues) and least square regression. In order to use lapack,
you must activate its usage by defining the following macros \ttcode{-DSTKUSELAPACK} at
compilation time and by linking your code with your installed lapack library using \ttcode{-llapack}
(at least in *nix operating systems).


\begin{table}[H]
\begin{tabular}{|l|l|l|}
\hline
Class                 & constructor                                 &  Note \\
\hline
lapack::Qr            & Qr( data, ref = false) & if data is an \ttcode{ArrayXX} and  \ttcode{ref} is true, \\
                      & Qr( data)                   & data will be overwritten by $Q$ \\
\hline
lapack::Svd           & Svd( data, ref = false, withU=true, withV=true) & if \ttcode{ref} is true, \\
                      & Svd(data)           & data will be overwritten by $Q$ \\
\hline
lapack::SymEigen      & SymEigen( data, ref = false) & if data is a \ttcode{SquareArray} and  \ttcode{ref} is true, \\
                      & SymEigen( data)                   & data will be overwritten by $Q$ \\
\hline
lapack::MultiLeastSquare & MultiLeastSquare( b, a, isBref = false, isBref=false) & \\
                      & MultiLeastSquare( b, a)           & \\
\hline
\end{tabular}
\end{table}

\textbf{To be continued...}

%-----------------------------------------
% bibliography
\bibliographystyle{plain}
\bibliography{rtkore}
\end{document}

